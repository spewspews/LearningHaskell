\documentclass{article}

\usepackage[paperwidth=5.5in,paperheight=8.5in,margin=0.5in,footskip=.25in]{geometry}
\usepackage{fontspec}
\usepackage{mathtools}
\usepackage{enumitem}
\usepackage{unicode-math}
\usepackage{fancyvrb}
\usepackage{syntax}
\usepackage{titlesec}

\DefineVerbatimEnvironment{code}{Verbatim}{baselinestretch=.8, samepage=true}

\titleformat{\section}{\scshape\Large}{\liningnums{\thesection}.\hspace{4pt}}{0em}{}
\titleformat{\subsection}{\scshape\large}{\liningnums{\thesubsection}.\hspace{4pt}}{0em}{}
\titlespacing{\section}{0em}{0em}{5pt}

\setmainfont{Garamond Premier Pro}[Contextuals=AlternateOff, Numbers=OldStyle]
\setmathfont{Libertinus Math}[Scale=MatchUppercase]
\setmonofont{JuliaMono}[Scale=0.7]

\setlength{\parindent}{1em}
\setlist{noitemsep}

\newcommand{\ttx}{\texttt}

\begin{document}
\begin{code}
module HuttonChap17 where
open import Haskell.Prelude
open import Haskell.Law.Equality using (sym; begin_; _≡⟨⟩_; step-≡; _∎; cong)
\end{code}

\section{Syntax and Semantics}

\begin{code}
data Expr : Set where
    Val : Int → Expr
    Add : Expr → Expr → Expr
{-# COMPILE AGDA2HS Expr #-}

eval : Expr → Int
eval (Val n) = n
eval (Add eₗ eᵣ) = eval eₗ + eval eᵣ
{-# COMPILE AGDA2HS eval #-}
\end{code}

\section{Adding a Stack}

\begin{code}
Stack = List Int
{-# COMPILE AGDA2HS Stack #-}

push : Int → Stack → Stack
push n s = n ∷ s
{-# COMPILE AGDA2HS push #-}

add : Stack → Stack
add [] = []
add (x ∷ []) = []
add (x ∷ y ∷ s) = y + x ∷ s
{-# COMPILE AGDA2HS add #-}

module DefineEval' where
  postulate
    eval' : Expr → Stack → Stack
    eval'≡eval : (e : Expr) → (s : Stack) → eval' e s ≡ eval e ∷ s

  eval'-val : (n : Int) → (s : Stack) → eval' (Val n) s ≡ push n s
  eval'-val n s =
    begin
      eval' (Val n) s
    ≡⟨ eval'≡eval (Val n) s ⟩ -- Specification
      eval (Val n) ∷ s
    ≡⟨⟩ -- Apply eval
      n ∷ s
    ≡⟨⟩ -- Unapply push
      push n s
    ∎
\end{code}
\begin{code}
  eval'-add : (x y : Expr) → (s : Stack)
    → eval' (Add x y) s ≡ add (eval' y (eval' x s))
\end{code}
\begin{code}
  eval'-add x y s =
    begin
      eval' (Add x y) s
    ≡⟨ eval'≡eval (Add x y) s ⟩ -- Specification
      eval (Add x y) ∷ s
    ≡⟨⟩ -- Apply eval
      eval x + eval y ∷ s
    ≡⟨⟩ -- Unapply add
      add (eval y ∷ eval x ∷ s)
    ≡⟨ cong (λ s → add (eval y ∷ s)) (sym (eval'≡eval x s)) ⟩ -- Induction
      add (eval y ∷ eval' x s)
    ≡⟨ cong add (sym (eval'≡eval y (eval' x s))) ⟩ -- Induction
      add (eval' y (eval' x s))
    ∎
\end{code}
\begin{code}
eval' : Expr → Stack → Stack
eval' (Val n) s = push n s
eval' (Add eₗ eᵣ) s = add (eval' eᵣ (eval' eₗ s))
{-# COMPILE AGDA2HS eval' #-}
\end{code}
\begin{code}
eval'≡eval : (e : Expr) → (s : Stack) → eval' e s ≡ eval e ∷ s
eval'≡eval (Val n) s = refl
eval'≡eval (Add x y) s =
  begin
    eval' (Add x y) s
  ≡⟨⟩ -- Apply eval'
    add (eval' y (eval' x s))
  ≡⟨ cong (λ s → add (eval' y s)) (eval'≡eval x s) ⟩ -- Induction
    add (eval' y (eval x ∷ s))
  ≡⟨ cong add (eval'≡eval y (eval x ∷ s)) ⟩ -- Induction
    add (eval y ∷ eval x ∷ s)
  ≡⟨⟩ -- Unapply add and eval
    eval (Add x y) ∷ s
  ∎
eval≡eval' : (e : Expr) → (s : Stack) → eval e ∷ s ≡ eval' e s
eval≡eval' e s = sym $ eval'≡eval e s
\end{code}

\noindent
Since \verb!eval' e s! is the same as \verb!eval e ∷ s!, this is evidence that \verb!eval' e s! is a non-empty list:

\begin{code}
open import Haskell.Prim using (NonEmpty; itsNonEmpty)
open import Haskell.Law.Equality using (subst)

instance
  eval'-nonempty : ⦃ e : Expr ⦄ → NonEmpty (eval' e [])
  eval'-nonempty ⦃ e ⦄ = subst NonEmpty (eval≡eval' e []) itsNonEmpty
\end{code}

\noindent
Since we know that \verb!eval' e []! is always non-empty, it is safe to apply head to it and get an equivalent definition of \texttt{eval}:

\begin{code}
eval¹ : ⦃ Expr ⦄ → Int
eval¹ ⦃ e ⦄ = head (eval' e [])
{-# COMPILE AGDA2HS eval¹ #-}
\end{code}

\section{Adding a Continuation}

\begin{code}
Cont = Stack → Stack
{-# COMPILE AGDA2HS Cont #-}

module DefineEval'' where
  postulate
    eval'' : Expr → Cont → Cont
    eval''≡eval' : (e : Expr) → (c : Cont) → (s : Stack)
      → eval'' e c s ≡ c (eval' e s)

  eval''-val : (n : Int) → (c : Cont) → (s : Stack)
    → eval'' (Val n) c s ≡ c (push n s)
  eval''-val n c s =
    begin
      eval'' (Val n) c s
    ≡⟨ eval''≡eval' (Val n) c s ⟩ -- Postulate
      c (eval' (Val n) s)
    ≡⟨⟩ -- Apply eval'
      c (push n s)
    ≡⟨⟩ -- Unapply ∘
      (c ∘ push n) s
    ∎
  eval''-add : (x y : Expr) → (c : Cont) → (s : Stack)
    → eval'' (Add x y) c s ≡ eval'' x (eval'' y (c ∘ add)) s
  eval''-add x y c s =
    begin
      eval'' (Add x y) c s
    ≡⟨ eval''≡eval' (Add x y) c s ⟩
      c (eval' (Add x y) s)
    ≡⟨⟩ -- Apply eval'
      c (add (eval' y (eval' x s)))
    ≡⟨⟩ -- Unapply ∘
      (c ∘ add) (eval' y (eval' x s))
    ≡⟨ sym (eval''≡eval' y (c ∘ add) (eval' x s)) ⟩ -- Induction y
      eval'' y (c ∘ add) (eval' x s)
    ≡⟨ sym (eval''≡eval' x (eval'' y (c ∘ add)) s) ⟩ -- Induction x
      eval'' x (eval'' y (c ∘ add)) s
    ∎
\end{code}
\begin{code}
eval'' : Expr → Cont → Cont
eval'' (Val n) c = c ∘ push n
eval'' (Add x y) c = eval'' x (eval'' y (c ∘ add))
{-# COMPILE AGDA2HS eval'' #-}
\end{code}
\begin{code}
eval''≡eval' : (e : Expr) → (c : Cont) → (s : Stack)
  → eval'' e c s ≡ c (eval' e s)
eval''≡eval' (Val x) c s = refl
\end{code}
\begin{code}
eval''≡eval' (Add x y) c s =
  begin
    eval'' (Add x y) c s
  ≡⟨⟩ -- Apply eval''
    eval'' x (eval'' y (c ∘ add)) s
  ≡⟨ eval''≡eval' x (eval'' y (c ∘ add)) s ⟩ -- Induction
    eval'' y (c ∘ add) (eval' x s)
  ≡⟨ eval''≡eval' y (c ∘ add) (eval' x s) ⟩ -- Induction
    (c ∘ add) (eval' y (eval' x s))
  ≡⟨⟩ -- Apply add
    c (eval' (Add x y) s)
  ∎
\end{code}

\noindent
Thus \texttt{eval'} is simply redefined as follows:

\begin{code}
eval'¹ : Expr → Cont
eval'¹ e = eval'' e id
{-# COMPILE AGDA2HS eval'¹ #-}
\end{code}

\section{Defunctionalising}

\begin{code}
haltC : Cont
haltC = id
{-# COMPILE AGDA2HS haltC #-}

pushC : Int → Cont → Cont
pushC n c = c ∘ push n
{-# COMPILE AGDA2HS pushC #-}

addC : Cont → Cont
addC c = c ∘ add
{-# COMPILE AGDA2HS addC #-}

data Code : Set where
  HALT : Code
  PUSH : Int → Code → Code
  ADD : Code → Code
{-# COMPILE AGDA2HS Code deriving Show #-}

exec : Code → Cont
exec HALT = haltC
exec (PUSH n c) = pushC n (exec c)
exec (ADD c) = addC (exec c)
{-# COMPILE AGDA2HS exec #-}

comp' : Expr → Code → Code
comp' (Val n) c = PUSH n c
comp' (Add x y) c = comp' x (comp' y (ADD c))
{-# COMPILE AGDA2HS comp' #-}

comp : Expr → Code
comp e = comp' e HALT
{-# COMPILE AGDA2HS comp #-}
\end{code}
\begin{code}
exec-comp'≡eval'' : (e : Expr) → (c : Code)
  → exec (comp' e c) ≡ eval'' e (exec c)
exec-comp'≡eval'' (Val n) c = refl
exec-comp'≡eval'' (Add x y) c =
  begin
    exec (comp' (Add x y) c)
  ≡⟨⟩ -- Apply comp'
    exec (comp' x (comp' y (ADD c)))
  ≡⟨ exec-comp'≡eval'' x (comp' y (ADD c)) ⟩ -- Induction
    eval'' x (exec (comp' y (ADD c)))
  ≡⟨ cong (eval'' x) (exec-comp'≡eval'' y (ADD c)) ⟩ -- Induction
    eval'' x (eval'' y (exec (ADD c)))
  ≡⟨⟩ -- Apply exec
    eval'' x (eval'' y (addC (exec c)))
  ≡⟨⟩ -- Unapply eval''
    eval'' (Add x y) (exec c)
  ∎
\end{code}
\begin{code}
exec-comp≡eval' : (e : Expr) → (s : Stack) → exec (comp e) s ≡ eval' e s
exec-comp≡eval' e s =
  begin
    exec (comp e) s
  ≡⟨⟩ -- Apply comp
    exec (comp' e HALT) s
  ≡⟨ cong (_$ s) (exec-comp'≡eval'' e HALT) ⟩
    eval'' e (exec HALT) s
  ≡⟨⟩ -- Apply exec
    eval'' e id s
  ≡⟨ eval''≡eval' e id s ⟩
    id (eval' e s)
  ≡⟨⟩ -- Apply id
    eval' e s
  ∎
\end{code}

\noindent
Alternatively, explicitly with lists:

\begin{code}
data Op : Set where
  PUSHOP : Int → Op
  ADDOP : Op
{-# COMPILE AGDA2HS Op #-}
\end{code}
\begin{code}
Prog = List Op
{-# COMPILE AGDA2HS Prog #-}
\end{code}
\begin{code}
execute : Prog → Cont
execute [] = haltC
execute (PUSHOP n ∷ os) = pushC n (execute os)
execute (ADDOP ∷ os) = addC (execute os)
{-# COMPILE AGDA2HS execute #-}
\end{code}
\begin{code}
compile' : Expr → Prog → Prog
compile' (Val n) p = PUSHOP n ∷ p
compile' (Add x y) p = compile' x (compile' y (ADDOP ∷ p))
{-# COMPILE AGDA2HS compile' #-}
\end{code}
\begin{code}
compile : Expr → Prog
compile e = compile' e []
{-# COMPILE AGDA2HS compile #-}
\end{code}
\begin{code}
execute-compile'≡eval'' : (e : Expr) → (p : Prog)
  → execute (compile' e p) ≡ eval'' e (execute p)
execute-compile'≡eval'' (Val n) p = refl
execute-compile'≡eval'' (Add x y) p =
  begin
    execute (compile' (Add x y) p)
  ≡⟨⟩ -- Apply compile'
    execute (compile' x (compile' y (ADDOP ∷ p)))
  ≡⟨ execute-compile'≡eval'' x (compile' y (ADDOP ∷ p)) ⟩ -- Induction
    eval'' x (execute (compile' y (ADDOP ∷ p)))
  ≡⟨ cong (eval'' x) (execute-compile'≡eval'' y (ADDOP ∷ p)) ⟩ -- Induction
    eval'' x (eval'' y (execute $ ADDOP ∷ p))
  ≡⟨⟩ -- Apply execute
    eval'' x (eval'' y (addC (execute p)))
  ≡⟨⟩ -- apply addC
    eval'' x (eval'' y ((execute p) ∘ add))
  ≡⟨⟩ -- Unapply eval''
    eval'' (Add x y) (execute p)
  ∎
\end{code}
\begin{code}
execute-compile≡eval' : (e : Expr) → (s : Stack)
  → execute (compile e) s ≡ eval' e s
execute-compile≡eval' e s =
  begin
    execute (compile e) s
  ≡⟨⟩ -- Apply compile
    execute (compile' e []) s
  ≡⟨ cong (_$ s) (execute-compile'≡eval'' e []) ⟩
    eval'' e (execute []) s
  ≡⟨⟩ -- Apply execute
    eval'' e haltC s
  ≡⟨ eval''≡eval' e haltC s ⟩
    haltC (eval' e s)
  ≡⟨⟩ -- Apply haltC ≡ id
    eval' e s
  ∎

execute-compile≡eval : (e : Expr) → (s : Stack)
  → execute (compile e) s ≡ eval e ∷ s
execute-compile≡eval e s =
  begin
    execute (compile e) s
  ≡⟨ execute-compile≡eval' e s ⟩
    eval' e s
  ≡⟨ eval'≡eval e s ⟩
    eval e ∷ s
  ∎
\end{code}

\section{Combining the Steps}
\end{document}
   
