\documentclass{article}

\usepackage[paperwidth=5.5in,paperheight=8.5in,margin=0.5in,footskip=.25in]{geometry}
\usepackage{fontspec}
\usepackage{mathtools}
\usepackage{enumitem}
\usepackage{unicode-math}
\usepackage{fancyvrb}
\usepackage{syntax}
\usepackage{tikz}

\DefineVerbatimEnvironment{code}{Verbatim}{baselinestretch=.8, samepage=true}

\setmainfont{Garamond Premier Pro}[Contextuals=AlternateOff, Numbers=OldStyle]
\setmathfont{Libertinus Math}[Scale=MatchUppercase]
\setmonofont{JuliaMono}[Scale=0.7]

\setlength{\parindent}{1em}
\setlist{noitemsep}

\newcommand{\ttx}{\texttt}

\begin{document}
\begin{code}
module HuttonChap16 where

open import Haskell.Prelude
open import Haskell.Law.Equality using (sym; begin_; _≡⟨⟩_; step-≡; _∎; cong)
open import Haskell.Law.Num.Def using (+-assoc; +-comm)
open import Haskell.Law.Num.Int using (iLawfulNumInt)
\end{code}

\noindent
\textsc{Induction on Numbers}

\noindent
Proving the first fact about replicate:

\begin{code}
replicate : {a : Set} → Nat → a → List a
replicate zero _ = []
replicate (suc n) x = x ∷ replicate n x

len-repl : {A : Set} → (n : Nat) → (x : A) → lengthNat (replicate n x) ≡ n
len-repl zero x = refl
len-repl (suc n) x =
  begin
    lengthNat (replicate (suc n) x)
  ≡⟨⟩ -- Apply replicate
    lengthNat (x ∷ replicate n x)
  ≡⟨⟩ -- Apply lengthNat
    suc (lengthNat (replicate n x))
  ≡⟨ cong suc (len-repl n x) ⟩
    suc n
  ∎
\end{code}

\noindent
Some facts about append:

\begin{code}
++-[] : {a : Set} → (xs : List a) → xs ++ [] ≡ xs
++-[] [] = begin ([] ++ []) ≡⟨⟩ [] ∎
++-[] (x ∷ xs) =
    begin
      (x ∷ xs) ++ []
    ≡⟨⟩ -- Apply ++
      x ∷ (xs ++ [])
    ≡⟨ cong (x ∷_) (++-[] xs) ⟩
      x ∷ xs
    ∎
\end{code}

\begin{code}
++-assoc : {a : Set} → (xs ys zs : List a)
    → (xs ++ ys) ++ zs ≡ xs ++ (ys ++ zs)
++-assoc [] ys zs =
    begin
      ([] ++ ys) ++ zs
    ≡⟨⟩ -- Apply ++
      ys ++ zs
    ≡⟨⟩ -- Unapply ++
      [] ++ (ys ++ zs)
    ∎
++-assoc (x ∷ xs) ys zs =
    begin
      ((x ∷ xs) ++ ys) ++ zs
    ≡⟨⟩ -- Apply ++
      (x ∷ (xs ++ ys)) ++ zs
    ≡⟨⟩ -- Apply ++
      x ∷ ((xs ++ ys) ++ zs)
    ≡⟨ cong (x ∷_) (++-assoc xs ys zs) ⟩
      x ∷ (xs ++ (ys ++ zs))
    ≡⟨⟩ -- Unapply ++
      (x ∷ xs) ++ (ys ++ zs)
    ∎
\end{code}

\noindent Hutton's example of elimination of append from flattening a tree:

\begin{code}
data Tree (a : Set) : Set where
    Leaf : a → Tree a
    Node : Tree a → Tree a → Tree a
{-# COMPILE AGDA2HS Tree #-}

flatten : {a : Set} → Tree a → List a
flatten (Leaf x) = x ∷ []
flatten (Node tl tr) = flatten tl ++ flatten tr
{-# COMPILE AGDA2HS flatten #-}

flatten' : {a : Set } → Tree a → List a → List a
flatten' (Leaf x) xs = x ∷ xs
flatten' (Node tₗ tᵣ) xs = flatten' tₗ (flatten' tᵣ xs)
{-# COMPILE AGDA2HS flatten' #-}
\end{code}

\begin{code}
flatten'-flatten : {a : Set} → (t : Tree a) → (xs : List a)
    → flatten' t xs ≡ flatten t ++ xs
flatten'-flatten (Leaf x) xs = refl
flatten'-flatten (Node tₗ tᵣ) xs =
  begin
    flatten' (Node tₗ tᵣ) xs
  ≡⟨⟩ -- Apply flatten'
    flatten' tₗ (flatten' tᵣ xs)
  ≡⟨ cong (flatten' tₗ) (flatten'-flatten tᵣ xs) ⟩
    flatten' tₗ (flatten tᵣ ++ xs)
  ≡⟨ flatten'-flatten tₗ (flatten tᵣ ++ xs) ⟩
    flatten tₗ ++ (flatten tᵣ ++ xs)
  ≡⟨ sym (++-assoc (flatten tₗ) (flatten tᵣ) xs) ⟩
    (flatten tₗ ++ flatten tᵣ) ++ xs
  ≡⟨⟩ -- Unapply flatten
    flatten (Node tₗ tᵣ) ++ xs
  ∎

flatten'-≡-flatten : {a : Set} → (t : Tree a)
    → flatten' t [] ≡ flatten t
flatten'-≡-flatten (Leaf x) = refl
flatten'-≡-flatten (Node tₗ tᵣ) =
  begin
    flatten' (Node tₗ tᵣ) []
  ≡⟨⟩ -- Apply flatten'
    flatten' tₗ (flatten' tᵣ [])
  ≡⟨ cong (flatten' tₗ) (flatten'-flatten tᵣ []) ⟩ -- Apply the above equality
    flatten' tₗ (flatten tᵣ ++ [])
  ≡⟨ flatten'-flatten tₗ (flatten tᵣ ++ []) ⟩ -- Apply it again
    flatten tₗ ++ (flatten tᵣ ++ [])
  ≡⟨ cong (flatten tₗ ++_) (++-[] (flatten tᵣ)) ⟩ -- Remove trailing []
    flatten tₗ ++ flatten tᵣ
  ≡⟨⟩ -- Unapply flatten
    flatten (Node tₗ tᵣ)
  ∎
\end{code}

\noindent
\textsc{Compiler correctness}

\begin{code}
data Expr : Set where
    Val : Int → Expr
    Add : Expr → Expr → Expr
{-# COMPILE AGDA2HS Expr #-}

eval : Expr → Int
eval (Val n) = n
eval (Add x y) = eval x + eval y
{-# COMPILE AGDA2HS eval #-}

Stack = List Int
{-# COMPILE AGDA2HS Stack #-}

data Op : Set where
    PUSH : Int → Op
    ADD : Op
{-# COMPILE AGDA2HS Op #-}
\end{code}

\begin{code}
Code = List Op
{-# COMPILE AGDA2HS Code #-}

exec : Code → Stack → Stack
exec [] s = s
exec (PUSH n ∷ c) s = exec c $ n ∷ s
exec (ADD ∷ c) (m ∷ n ∷ s) = exec c $ n + m ∷ s
exec (ADD ∷ c) _ = []
{-# COMPILE AGDA2HS exec #-}

comp : Expr → Code → Code
comp (Val n) c = PUSH n ∷ c
comp (Add x y) c = comp x $ comp y $ ADD ∷ c
{-# COMPILE AGDA2HS comp #-}

comp-exec-eval : (e : Expr) → (c : Code) → (s : Stack)
    → exec (comp e c) s ≡ exec c (eval e ∷ s)
comp-exec-eval (Val n) c s =
  begin
    exec (comp (Val n) c) s
  ≡⟨⟩ -- Apply comp
    exec (PUSH n ∷ c) s
  ≡⟨⟩ -- Apply exec
    exec c (n ∷ s)
  ≡⟨⟩ -- Unapply eval
    exec c (eval (Val n) ∷ s)
  ∎
comp-exec-eval (Add x y) c s =
  begin
    exec (comp (Add x y) c) s
  ≡⟨⟩ -- Apply comp
    exec (comp x $ comp y $ ADD ∷ c) s
  ≡⟨ comp-exec-eval x (comp y $ ADD ∷ c) s ⟩ -- Induction
    exec (comp y $ ADD ∷ c) (eval x ∷ s)
  ≡⟨ comp-exec-eval y (ADD ∷ c) (eval x ∷ s) ⟩ -- Induction Again
    exec (ADD ∷ c) (eval y ∷ eval x ∷ s)
  ≡⟨⟩ -- Apply exec
    exec c ((eval x) + (eval y) ∷ s)
  ≡⟨⟩ -- Unapply eval
    exec c (eval (Add x y) ∷ s)
  ∎

compile : Expr → Code
compile e = comp e []
{-# COMPILE AGDA2HS compile #-}

compile-exec-eval : (e : Expr) → exec (compile e) [] ≡ eval e ∷ []
compile-exec-eval e =
  begin
    exec (compile e) []
  ≡⟨⟩ -- Apply compile
    exec (comp e []) []
  ≡⟨ comp-exec-eval e [] [] ⟩
    exec [] (eval e ∷ [])
  ≡⟨⟩ -- Apply exec
    eval e ∷ []
  ∎
\end{code}

\noindent
\textsc{Exercise 1.} Show that \verb!add n (Suc m) = Suc (add n m)! by induction on \texttt{n}

\begin{code}
+-suc : (n m : Nat) → n + (suc m) ≡ suc (n + m)
+-suc zero m = refl
+-suc (suc n) m =
  begin
    (suc n) + (suc m)
  ≡⟨⟩ -- Apply +
    suc (n + suc m)
  ≡⟨ cong suc (+-suc n m) ⟩
    suc (suc (n + m))
  ≡⟨⟩ -- Unapply +
    suc (suc n + m)
  ∎
\end{code}

\noindent
\textsc{Exercise 2.} Using this property, together with \texttt{add n zero = n}, show that addition is commutative, \texttt{add n m = add m n}, by induction on \texttt{n}.

\begin{code}
+-zero : (n : Nat) → n + zero ≡ n
+-zero zero = refl
+-zero (suc n) =
  begin
    suc n + zero
  ≡⟨⟩ -- Apply +
    suc (n + zero)
  ≡⟨ cong suc (+-zero n) ⟩
    suc n
  ∎
+-commut : (n m : Nat) → n + m ≡ m + n
+-commut zero m =
  begin
    zero + m
  ≡⟨⟩ -- Apply +
    m
  ≡⟨ sym (+-zero m) ⟩
    m + zero
  ∎
+-commut (suc n) m =
  begin
    suc n + m
  ≡⟨⟩ -- Apply +
    suc (n + m)
  ≡⟨ cong suc (+-commut n m) ⟩
    suc (m + n)
  ≡⟨ sym (+-suc m n) ⟩
    m + suc n
  ∎
\end{code}

\noindent
\textsc{Exercise 3.}
Complete the proof of the correctness of replicate by showing that it produces a list with identical elements, \verb!all (== x) (replicate n x)!, by induction on $n ≥ 0$. Hint: show that the property is always \verb!True!.

\begin{code}
open import Haskell.Law.Eq.Def using (IsLawfulEq; eqReflexivity)
all-repl : ⦃ iEq : Eq a ⦄ → ⦃ IsLawfulEq a ⦄ → (n : Nat) → (x : a)
    → all (_== x) (replicate n x) ≡ True
all-repl zero x = refl
all-repl (suc n) x =
  begin
    all (_== x) (replicate (suc n) x)
  ≡⟨⟩ -- Apply replicate
    all (_== x) (x ∷ replicate n x)
  ≡⟨⟩ -- Apply all
    (x == x) && (all (_== x) (replicate n x))
  ≡⟨ cong ((x == x) &&_) (all-repl n x) ⟩ -- Induction
    (x == x) && True
  ≡⟨ cong (_&& True) (eqReflexivity x) ⟩ -- Reflexivity x == x
    True
  ∎
\end{code}

\noindent
\textsc{Exercise 4.} This is \verb!++-[]! and \verb!++-assoc! above.
\vspace{3pt}

\noindent
\textsc{Exercise 5.} Using the above definition for \texttt{++}, together with the definitions for \texttt{take} and \texttt{drop} show that \verb!take n xs ++ drop n xs! $=$ \verb!xs!, by simultaneous induction on the integer n and the list xs. Hint: there are three cases, one for each pattern of arguments in the definitions of take and drop.

\begin{code}
take-drop-nat : {a : Set} → (n : Nat) → (xs : List a)
    → takeNat n xs ++ dropNat n xs ≡ xs
take-drop-nat n [] = refl
take-drop-nat zero (x ∷ xs) =
  begin
    takeNat zero (x ∷ xs) ++ dropNat zero (x ∷ xs)
  ≡⟨⟩ -- Apply takeNat and dropNat
    [] ++ x ∷ xs
  ≡⟨⟩
    x ∷ xs
  ∎
take-drop-nat (suc n) (x ∷ xs) =
  begin
    takeNat (suc n) (x ∷ xs) ++ dropNat (suc n) (x ∷ xs)
  ≡⟨⟩ -- Apply takeNat and dropNat and ++
    x ∷ takeNat n xs ++ dropNat n xs
  ≡⟨ cong (x ∷_) (take-drop-nat n xs) ⟩
    x ∷ xs
  ∎
take-drop : {a : Set} → (n : Int) → ⦃ iNN : IsNonNegativeInt n ⦄
    → (xs : List a) → take n xs ++ drop n xs ≡ xs
take-drop n xs =
  begin
    take n xs ++ drop n xs
  ≡⟨⟩ -- Apply take and drop
    takeNat (intToNat n) xs ++ dropNat (intToNat n) xs
  ≡⟨ take-drop-nat (intToNat n) xs ⟩
    xs
  ∎
\end{code}

\noindent
\textsc{Exercise 6.} Given the \texttt{Tree} definition above, show that the number of leaves in such a tree is always one greater than the number of nodes, by induction on trees. Hint: start by defining functions that count the number of leaves and nodes in a tree.

\begin{code}
nLeaves : {a : Set} → Tree a → Int
nLeaves (Leaf x) = 1
nLeaves (Node tₗ tᵣ) = nLeaves tₗ + nLeaves tᵣ
{-# COMPILE AGDA2HS nLeaves #-}

nNodes : {a : Set} → Tree a → Int
nNodes (Leaf x) = 0
nNodes (Node tₗ tᵣ) = 1 + nNodes tₗ + nNodes tᵣ
{-# COMPILE AGDA2HS nNodes #-}

leaves-nodes : {a : Set} → (t : Tree a)
    → nLeaves t ≡ 1 + nNodes t
leaves-nodes (Leaf x) = refl
leaves-nodes (Node tₗ tᵣ) =
  begin
    nLeaves (Node tₗ tᵣ)
  ≡⟨⟩
    nLeaves tₗ + nLeaves tᵣ
  ≡⟨ cong (_+ (nLeaves tᵣ)) (leaves-nodes tₗ) ⟩
    1 + nNodes tₗ + nLeaves tᵣ
  ≡⟨ cong ((1 + nNodes tₗ) +_) (leaves-nodes tᵣ) ⟩
    1 + nNodes tₗ + (1 + nNodes tᵣ)
  ≡⟨ +-assoc 1 (nNodes tₗ) (1 + nNodes tᵣ) ⟩
    1 + (nNodes tₗ + (1 + nNodes tᵣ))
  ≡⟨ cong (1 +_) (sym (+-assoc (nNodes tₗ) 1 (nNodes tᵣ))) ⟩
    1 + (nNodes tₗ + 1 + nNodes tᵣ)
  ≡⟨ cong (1 +_) (cong (_+ nNodes tᵣ) (+-comm (nNodes tₗ) 1)) ⟩
    1 + (1 + nNodes tₗ + nNodes tᵣ)
  ≡⟨⟩
    1 + nNodes (Node tₗ tᵣ)
  ∎
\end{code}

\noindent
\textsc{Exercise 7.} Verify the functor laws for the \texttt{Maybe} type. Hint: the proofs proceed by case analysis, and do not require the use of induction.

\begin{code}
module FunctorLawsMaybe where
  identity : {a : Set} → (m : Maybe a) → (fmap id) m ≡ id m
  identity Nothing =
    begin
      fmap id Nothing
    ≡⟨⟩ -- Apply fmap
      Nothing
    ≡⟨⟩ -- Unapply id
      id Nothing
    ∎
  identity (Just x) =
    begin
      fmap id (Just x)
    ≡⟨⟩ -- Apply fmap
      Just (id x)
    ≡⟨⟩ -- Apply id
      Just x
    ≡⟨⟩ -- Unapply id
      id (Just x)
    ∎
  composition : {a b c : Set}
    → (m : Maybe a) → (f : a → b) → (g : b → c)
    → fmap (g ∘ f) m ≡ (fmap g ∘ fmap f) m
  composition Nothing f g =
      begin
        fmap (g ∘ f) Nothing
      ≡⟨⟩ -- Apply fmap
        Nothing
      ≡⟨⟩ -- Unapply fmap
        fmap g Nothing
      ≡⟨⟩ -- Unapply fmap
        fmap g (fmap f Nothing)
      ≡⟨⟩ -- Unapply ∘
        (fmap g ∘ fmap f) Nothing
      ∎
  composition (Just x) f g =
    begin
      fmap (g ∘ f) (Just x)
    ≡⟨⟩ -- Apply fmap
      Just ((g ∘ f) x)
    ≡⟨⟩ -- Apply ∘
      Just (g (f x))
    ≡⟨⟩ -- Unapply fmap
      fmap g (Just (f x))
    ≡⟨⟩ -- Unapply fmap
      fmap g (fmap f (Just x))
    ≡⟨⟩ -- Unapply ∘
      (fmap g ∘ fmap f) (Just x)
    ∎
module LawfulFunctorMaybe where
  open import Haskell.Law.Functor.Def
    using (IsLawfulFunctor; identity; composition)
  instance
    isLawful : IsLawfulFunctor Maybe
    identity ⦃ isLawful ⦄ = FunctorLawsMaybe.identity
    composition ⦃ isLawful ⦄ = FunctorLawsMaybe.composition
\end{code}

\noindent
\textsc{Exercise 8.} Given the instance declaration below, verify the functor laws for the \texttt{Tree} type, by induction on trees.

\begin{code}
open import Haskell.Prim.Functor using (DefaultFunctor)
treeMap : {a b : Set} → (a → b) → (Tree a) → (Tree b)
treeMap f (Leaf x) = Leaf (f x)
treeMap f (Node tₗ tᵣ) = Node (treeMap f tₗ) (treeMap f tᵣ)
{-# COMPILE AGDA2HS treeMap #-}
dft : DefaultFunctor Tree
dft = record { fmap = treeMap }
instance
  iFunctorTree : Functor Tree
  iFunctorTree = record { DefaultFunctor dft }
  {-# COMPILE AGDA2HS iFunctorTree #-}
module FunctorLawsTree where
  identity : (t : Tree a) → (fmap id) t ≡ id t
  identity (Leaf x) = refl
  identity (Node tₗ tᵣ) =
    begin
      fmap id (Node tₗ tᵣ)
    ≡⟨⟩ -- Apply fmap
      Node (fmap id tₗ) (fmap id tᵣ)
    ≡⟨ cong (λ x → Node x (fmap id tᵣ)) (identity tₗ) ⟩
      Node (id tₗ) (fmap id tᵣ)
    ≡⟨ cong (Node (id tₗ)) (identity tᵣ) ⟩
      Node (id tₗ) (id tᵣ)
    ≡⟨⟩ -- Apply and unapply id
      id (Node tₗ tᵣ)
    ∎
  composition : (t : Tree a) → (f : a → b) → (g : b → c)
    → fmap (g ∘ f) t ≡ (fmap g ∘ fmap f) t
  composition (Leaf x) f g = refl
  composition (Node tₗ tᵣ) f g =
    begin
      fmap (g ∘ f) (Node tₗ tᵣ)
    ≡⟨⟩ -- Apply fmap
      Node (fmap (g ∘ f) tₗ) (fmap (g ∘ f) tᵣ)
    ≡⟨ cong (λ x → Node x (fmap (g ∘ f) tᵣ)) (composition tₗ f g) ⟩
      Node ((fmap g ∘ fmap f) tₗ) (fmap (g ∘ f) tᵣ)
    ≡⟨ cong (Node ((fmap g ∘ fmap f) tₗ)) (composition tᵣ f g) ⟩
      Node ((fmap g ∘ fmap f) tₗ) ((fmap g ∘ fmap f) tᵣ)
    ≡⟨⟩ -- Unapply fmap
      fmap g (Node (fmap f tₗ) (fmap f tᵣ))
    ≡⟨⟩ -- Unapply fmap
      fmap g (fmap f (Node tₗ tᵣ))
    ≡⟨⟩ -- Unapply ∘
      (fmap g ∘ fmap f) (Node tₗ tᵣ)
    ∎
module LawfulFunctorTree where
  open import Haskell.Law.Functor.Def
    using (IsLawfulFunctor; identity; composition)
  instance
    isLawful : IsLawfulFunctor Tree
    identity ⦃ isLawful ⦄ = FunctorLawsTree.identity
    composition ⦃ isLawful ⦄ = FunctorLawsTree.composition
\end{code}

\noindent
\textsc{Exercise 9.} Verify the applicative laws for the \texttt{Maybe} type.

\begin{code}
module ApplicativeLawsMaybe where
  identity : {a : Set} → (m : Maybe a) → (pure id <*> m) ≡ m
  identity Nothing =
    begin
      pure id <*> Nothing
    ≡⟨⟩ -- Apply pure and <*>
      Nothing
    ∎
  identity (Just x) =
    begin
      pure id <*> Just x
    ≡⟨⟩ -- Apply pure
      Just id <*> Just x
    ≡⟨⟩ -- Apply <*>
      Just (id x)
    ≡⟨⟩ -- Apply id
      Just x
    ∎

  composition : {a b c : Set}
    → (x : Maybe (b → c)) → (y : Maybe (a → b)) → (z : Maybe a)
    → (pure _∘_ <*> x <*> y <*> z) ≡ (x <*> (y <*> z))
  composition Nothing y z =
    begin
      pure _∘_ <*> Nothing <*> y <*> z
    ≡⟨⟩ -- Apply pure and <*>
      Nothing <*> y <*> z
    ≡⟨⟩ -- Apply the rest of the <*>
      Nothing
    ≡⟨⟩ -- Unapply <*> on the right
      Nothing <*> (y <*> z)
    ∎
  composition (Just x) Nothing z =
    begin
      pure _∘_ <*> Just x <*> Nothing <*> z
    ≡⟨⟩ -- Apply pure and <*>
      Just (x ∘_) <*> Nothing <*> z
    ≡⟨⟩ -- Apply <*>
      Nothing <*> z
    ≡⟨⟩ -- Apply <*>
      Nothing
    ≡⟨⟩ -- Unapply <*>
      Nothing <*> z
    ≡⟨⟩ -- Unapply <*>
      Just x <*> (Nothing <*> z)
    ∎
  composition (Just x) (Just y) Nothing =
    refl -- Same kind of proof as above.
\end{code}

\begin{code}
  composition (Just x) (Just y) (Just z) =
    begin
      pure _∘_ <*> Just x <*> Just y <*> Just z
    ≡⟨⟩ -- Apply pure and <*>
      Just (x ∘_) <*> Just y <*> Just z
    ≡⟨⟩ -- Apply <*>
      Just (x ∘ y) <*> Just z
    ≡⟨⟩ -- Apply <*>
      Just ((x ∘ y) z)
    ≡⟨⟩ -- Apply ∘
      Just (x (y z))
    ≡⟨⟩ -- Unapply <*>
      Just x <*> Just (y z)
    ≡⟨⟩ -- Unapply <*>
      Just x <*> (Just y <*> Just z)
    ∎

  homomorphism : {a b : Set} → (f : a → b) (x : a)
    → ((pure f) <*> (pure x)) ≡ (pure (f x))
  homomorphism f x =
    begin
      pure f <*> pure x
    ≡⟨⟩ -- Apply pure
      Just f <*> Just x
    ≡⟨⟩ -- Apply <*>
      Just (f x)
    ≡⟨⟩ -- Unapply pure
      pure (f x)
    ∎

  interchange : {a b : Set} → (x : Maybe (a → b)) (y : a)
    → (x <*> (pure y)) ≡ (pure (λ f → f y) <*> x)
  interchange Nothing y =
    begin
      Nothing <*> pure y
    ≡⟨⟩ -- Apply <*>
      Nothing
    ≡⟨⟩ -- Unapply <*>
      pure (_$ y) <*> Nothing
    ∎
  interchange (Just x) y =
    begin
      (Just x) <*> pure y
    ≡⟨⟩ -- Apply <*>
      Just (x y)
    ≡⟨⟩ -- Unapply $
      Just ((_$ y) x)
    ≡⟨⟩ -- Unapply <*>
      pure (_$ y) <*> Just x
    ∎
\end{code}

\begin{code}
module LawfulApplicativeMaybe where
  open import Haskell.Law.Applicative.Def
    using (IsLawfulApplicative; identity; composition;
      homomorphism; interchange; functor)
  instance
    isLawful : IsLawfulApplicative Maybe
    identity ⦃ isLawful ⦄ = ApplicativeLawsMaybe.identity
    composition ⦃ isLawful ⦄ = ApplicativeLawsMaybe.composition
    homomorphism ⦃ isLawful ⦄ = ApplicativeLawsMaybe.homomorphism
    interchange ⦃ isLawful ⦄ = ApplicativeLawsMaybe.interchange
    functor ⦃ isLawful ⦄ f Nothing = refl -- These are by definition.
    functor ⦃ isLawful ⦄ f (Just x) = refl
\end{code}

\noindent
\textsc{Exercise 10.} Verify the monad laws for the list type. Hint: the proofs can be completed using simple properties of list comprehensions.

\begin{code}
module MonadLawsList where
  leftIdentity : {a : Set} → (x : a) (f : a → List b)
    → ((return x) >>= f) ≡ f x
  leftIdentity x f =
    begin
      (return x) >>= f
    ≡⟨⟩ -- Apply return
      (x ∷ []) >>= f
    ≡⟨⟩ -- Apply >>=
      f x ++ []
    ≡⟨ ++-[] (f x) ⟩
      f x
    ∎
\end{code}
\begin{code}
  fmap2bind : {a b : Set} → (f : a → b) → (xs : List a)
    → fmap f xs ≡ (xs >>= (return ∘ f))
  fmap2bind f [] = refl
  fmap2bind f (x ∷ xs) =
    begin
      fmap f (x ∷ xs)
    ≡⟨⟩ -- Apply fmap
      (f x ∷ []) ++ fmap f xs
    ≡⟨ cong ((f x ∷ []) ++_) (fmap2bind f xs) ⟩
      (f x ∷ []) ++ (xs >>= (return ∘ f))
    ≡⟨⟩ -- Unapply return
      (return (f x)) ++ (xs >>= (return ∘ f))
    ≡⟨⟩ -- Unapply ∘
      (return ∘ f) x ++ (xs >>= (return ∘ f))
    ≡⟨⟩ -- Unapply >>=
      (x ∷ xs) >>= (return ∘ f)
    ∎
\end{code}
\begin{code}
  import Haskell.Law.Functor as Functor
  rightIdentity : {a : Set} → (xs : List a) → (xs >>= return) ≡ xs
  rightIdentity xs =
    begin
      xs >>= return
    ≡⟨⟩ -- Unapply id
      xs >>= (return ∘ id)
    ≡⟨ sym (fmap2bind id xs) ⟩
      map id xs
    ≡⟨ Functor.identity xs ⟩
      xs
    ∎
\end{code}
\begin{code}
  ++-distrib : {a : Set} → (xs ys : List a) → (f : a → List b)
    → ((xs >>= f) ++ (ys >>= f)) ≡ ((xs ++ ys) >>= f)
  ++-distrib [] ys f = refl
  ++-distrib (x ∷ xs) ys f =
    begin
      ((x ∷ xs) >>= f) ++ (ys >>= f)
    ≡⟨⟩ -- Apply >>=
      (f x ++ (xs >>= f)) ++ (ys >>= f)
    ≡⟨ ++-assoc (f x) (xs >>= f) (ys >>= f) ⟩
      f x ++ (xs >>= f) ++ (ys >>= f)
    ≡⟨ cong (f x ++_) (++-distrib xs ys f) ⟩
      f x ++ ((xs ++ ys) >>= f)
    ≡⟨⟩ -- Unapply >>=
      (x ∷ (xs ++ ys)) >>= f
    ≡⟨⟩ -- Unapply ++
      ((x ∷ xs) ++ ys) >>= f
    ∎
\end{code}
\begin{code}
  associativity : {a b c : Set}
    → (xs : List a) → (f : a → List b) → (g : b → List c)
    → (xs >>= λ x → f x >>= g) ≡ ((xs >>= f) >>= g)
  associativity [] f g = refl
  associativity (x ∷ xs) f g =
    begin
      (x ∷ xs) >>= (λ x → f x >>= g)
    ≡⟨⟩ -- Apply >>=
      (f x >>= g) ++ (xs >>= λ x → f x >>= g)
    ≡⟨ cong ((f x >>= g) ++_) (associativity xs f g) ⟩
      (f x >>= g) ++ ((xs >>= f) >>= g)
    ≡⟨ ++-distrib (f x) (xs >>= f) g ⟩
      (f x ++ (xs >>= f)) >>= g
    ≡⟨⟩ -- Unapply inner >>=
      ((x ∷ xs) >>= f) >>= g
    ∎
\end{code}
\begin{code}
  sequence2bind : {a b : Set}
    → (fs : List (a → b)) → (xs : List a)
    → (fs <*> xs) ≡ (fs >>= λ f → (xs >>= (return ∘ f)))
  sequence2bind [] xs = refl
  sequence2bind (f ∷ fs) xs =
    begin
      f ∷ fs <*> xs
    ≡⟨⟩ -- Apply <*>
      fmap f xs ++ (fs <*> xs)
    ≡⟨ cong (_++ (fs <*> xs)) (fmap2bind f xs) ⟩
      xs >>= (return ∘ f) ++ (fs <*> xs)
    ≡⟨ cong (xs >>= (return ∘ f) ++_) (sequence2bind fs xs) ⟩
      xs >>= (return ∘ f)
        ++ (fs >>= λ f → (xs >>= (return ∘ f)))
    ≡⟨⟩ -- Unapply λ
      ((λ f → (xs >>= (return ∘ f))) f)
        ++ (fs >>= λ f → (xs >>= (return ∘ f)))
    ≡⟨⟩ -- Unapply >>=
      (f ∷ fs) >>= (λ f → (xs >>= (return ∘ f)))
    ∎
\end{code}
\begin{code}
  rSequence2rBind : {a b : Set} → (xs : List a) → (ys : List b)
    → (xs *> ys) ≡ (xs >> ys)
  rSequence2rBind [] ys = refl
  rSequence2rBind (x ∷ xs) ys =
    begin
      (x ∷ xs) *> ys
    ≡⟨⟩ -- Apply *>
      (const id x) ∷ (fmap (const id) xs) <*> ys
    ≡⟨⟩ -- Apply <*>
      fmap (const id x) ys ++ (fmap (const id) xs <*> ys)
    ≡⟨⟩ -- Apply const
      fmap id ys ++ (fmap (const id) xs <*> ys)
    ≡⟨ cong (_++ (fmap (const id) xs <*> ys)) (Functor.identity ys) ⟩
      ys ++ (fmap (const id) xs <*> ys)
    ≡⟨⟩ -- Unapply *>
      ys ++ (xs *> ys)
    ≡⟨ cong (ys ++_) (rSequence2rBind xs ys) ⟩
      ys ++ (xs >> ys)
    ≡⟨⟩ -- Apply >>
      ys ++ (xs >>= λ x → ys)
    ≡⟨⟩ -- Unapply λ
      (λ x → ys) x ++ (xs >>= λ x → ys)
    ≡⟨⟩ -- Unapply >>=
      (x ∷ xs) >>= (λ x → ys)
    ≡⟨⟩ -- Unapply >>
      (x ∷ xs) >> ys
    ∎
\end{code}
\begin{code}
module LawfulMonadList where
  open import Haskell.Law.Applicative.List
  open import Haskell.Law.Monad.Def
    using (IsLawfulMonad; leftIdentity; rightIdentity;
      associativity; pureIsReturn; sequence2bind; fmap2bind;
      rSequence2rBind)
\end{code}
\begin{code}
  instance
    isLawful : IsLawfulMonad List
    leftIdentity ⦃ isLawful ⦄ = MonadLawsList.leftIdentity
    rightIdentity ⦃ isLawful ⦄ = MonadLawsList.rightIdentity
    associativity ⦃ isLawful ⦄ = MonadLawsList.associativity
    pureIsReturn ⦃ isLawful ⦄ _ = refl -- By definition
    sequence2bind ⦃ isLawful ⦄ = MonadLawsList.sequence2bind
    fmap2bind ⦃ isLawful ⦄ = MonadLawsList.fmap2bind
    rSequence2rBind ⦃ isLawful ⦄ = MonadLawsList.rSequence2rBind
\end{code}

\end{document}
